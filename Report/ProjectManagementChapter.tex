\chapter{Project Management}
\section{Development process}

The customer favoured the use of an iterative approach to the development process, where every sprint added a new layer of functionality, either to the application or the underlying model. Each sprint lasted for a period of 1-2 weeks, and the exact content was worked out in collaboration with the customer. Short term plans were favored over longer plans, due to the flexibility provided. While this made formulating definite goals for the final product difficult, the customer and the group were in agreement that due to the research intensive nature of the project, a high degree of flexibility was required. 

It was decided by the developers that they would have online meetings twice a week and an offline meeting once a week. 
The working hours was set to not less than 20 hours a week, but the developers was free to choose when to work themselves.

\section{Team Roles and Organization}
There was not much place for specific roles among the group, as it was a small group, and the project requiring that all the members are capable and willing to work at all the tasks that needed doing. Roles that was set for the Group was therefore mainly organizers, so that one person was to keep awareness of what work needed to be completed in a particular domain:
\begin{description}

\item[Organizer] Elias was made organizer, and getting the responsibilities of reminding the group of what to do.
\item[Documenter] Johannes was tasked with organizing documentation and distributing the work of writing the report to the group. 
\end{description}
 
