
\section{Translations}

The localization was implemented as described above, with several XML files containing the string values that was to be used in the GUI.  Which language the values are appropriate for is determined by the folder name and the active language in the android device. If a folder and string value for the current language is not found by android, the default values will be used instead. This application has the default values (for all the GUI text) written in English, as that was believed to be a more widely understood language than Norwegian. If a user has the phone set to use any Norwegian language (nynorsk or bokmål), the phone will instead display all the applications text in Norwegian. To accomplish this, it was critical that all text that would show up on-screen would only be retrieved from these string files, and be stored as a code or reference where this was not possible to do directly. The database storing all the messages the user has received is the most important example of this, as it only contains a code that enables the application to determine which message should be displayed. The application will then retrieve the appropriate string, while the android operation system will decide which language file to retrieve the string from. This means that process to translate the application was quite simple, as all that was needed was to copy a few files into a new folder with the appropriate name, and rewrote the content of the string to something appropriate for the new language. Any developers who would like to alter the application to display in a different language, would only have to perform the same process. \\
The application is localized for Norwegian and English, although all that is needed for localization an knowledge of English, and the language the strings are to be translated into. 