\section{Falling: Causes, Consequences and how to avoid it}
This report collected information from several sources on the subject of falls among the elderly. Particular focus is given to risks and avoidance strategies.

\subsection{Risk factors}
There are several factors that predicate risk of falling. Typical risk factors are:
\begin{itemize}
\item 
\textbf{Biological risk factors (sorted by order of significance):  \cite{fallsRubenstein}}
\begin{itemize}
\item Muscle weakness
\item Balance deficit
\item Gait deficit
\item Visual deficit
\item Mobility limitation
\item Cognitive impairment
\item Impaired functional status
\item Postural hypertension
\end{itemize}
\item 
\textbf{Behavioral risk factors(Not in sorted order):}
\begin{itemize}
\item Inactivity \cite{cdcComProg}
\item Medications \cite{cdcComProg}
\item Alcohol use \cite{cdcComProg}
\item Living alone \cite{housing}
\end{itemize}
\item 
\textbf{Home/environmental risk factors(not in sorted order): \cite{WHO}}
\begin{itemize}
\item Bad footwear or clothing.
\item Dangers in the house or in public places.
\item Unfamiliarity with walking aids such as canes, crutches, walking chairs.
\end{itemize}
\end{itemize}

A fall is normally caused by the interaction of two or more of these risk factors, but home or environmental risk factors cause only 30-50\% of all falls\cite{fallsRubenstein, cdcComProg}. What this means is that more than half of all falls happen without any influence of environmental factors. Also, in most of these cases where external factors play a role, the fall is in reality caused by an interaction between these and physiological aspects. The second and third most common single causes of falling are gait/balance disorders and dizziness/vertigo, followed by \emph{drop attacks}, which are sudden falls without loss of consciousness or dizziness\cite{fallsRubenstein}. As the project task is to develop a model for physical movement, it might be necessary to down-prioritize identifying risk for dizziness/vertigo and drop attacks, as well as ignore external factors, in order to focus purely on physiological aspects, such as gait or balance.

Many older adults are unaware of their risk factors, and therefore unable to take preventive actions. Even older adults with a history of falling have normally been given little education about the potential risk factors. Any sort of risk assessment, is therefore be very beneficial, especially when the results are discussed with a healthcare provider\cite{cdcComProg}. This fact illustrates the usefulness of the planned application. 

It has been shown that many of the biological risk factors can be reduced effectively by preforming regular physical activities. Specifically strength, gait and balance has been shown to be improvable by simple exercise regimes - even for frail patients - in a number of studies\cite{LMTassessPrev, cdcComProg, WHO}. A potential focus area for the app can therefore be encouraging exercise and healthy lifestyles.

\subsection{Advice for prevention}
The individual can easily reduce the risk of falling greatly by taking certain measures. A list of measures normally, by the authorities, recommended for older people in the target group is given below\cite{cdcYouPrevent}:

\begin{itemize}
\item Begin a regular exercise program.
\begin{itemize}
\item Exercises that improve balance and coordination are the most helpful.
\item Is the only measure that by itself reduces the risk of falling independently of individual circumstances. 
\end{itemize}
\item Have your health care provider review your medicines, even over-the-counter ones.
\begin{itemize}
\item Avoid medicines that can make you dizzy or sleepy.
\end{itemize}
\item Have your vision checked.
\item Make your home safer.
\begin{itemize}
\item Remove or fasten small rugs and carpets.
\item Remove wires and cords away from commonly taken paths. Tape wires to the walls, and if necessary install an additional power outlet.
\item Keep items where you can reach them without having to climb. If you have to use a step stool, get a stable one with handrails.
\item Remove loose items that can make you trip (books, papers, clothes, etc.) from the floor and stairs.
\item Add grab bars and non-slip mats to the bathroom.
\item Make your home brighter.
\item Repair broken or uneven steps and handrails in the staircase.
\item Avoid using the staircase more than necessary, for instance by installing a light switch at the top as well as the bottom of the stairs. 
\item Wear shoes, even inside. Avoid walking barefoot or wearing slippers.
\end{itemize}
\end{itemize}

\subsection{Physiological aspects}

The list below explains the physiological factors that contribute to stability. “A marked deficit in any one of these factors may be sufficient to increase the risk of falling; however, a combination of mild or moderate impairments in multiple physiological domains also may increase the risk of falling. By directly assessing an individual's physiological abilities, intervention strategies can be implemented to target areas of deficit.”\footnote{"A physiological Profile Approach to Falls Risk Assessment and Prevention", Stephen R. Lord,Hylton B. Menz and Anne Tiedemann, Journal of the American Physical Therapy Association. } \cite{LMTassessPrev}

\begin{itemize}
\item Reaction time
\begin{itemize}
\item Hand
\item Foot
\end{itemize}
\item Vision
\begin{itemize}
\item Contrast sensitivity
\item Visual acuity
\end{itemize}
\item Vestibular function
\begin{itemize}
\item Visual field dependence
\end{itemize}
\item 
Peripheral sensation
\begin{itemize}
\item Tactile sensitivity
\item Vibration sense
\item Proprioception
\end{itemize}
\item Muscle force
\begin{itemize}
\item Knee flexion
\item Knee extension
\item Ankle dorsiflection
\end{itemize}
\end{itemize}


There exists an array of tests that can measure the performance on these aspects. Lord et al. provides one possible set of tests. Some of these are possible to implement in an app, but none of them are based on hip movement.

The test developed by Lord et al. Was used to classify older adults into fallers or non-fallers, and has an accuracy of $75-80\%$ in different experiments.\cite{LMTassessPrev} If a test which disregards hip movement patterns performs well, there may be reasons to believe that hip movement is only a minor factor in detecting fall risk. However, 

Walking behaviour seems to be generally associated with falling. Lord et al. Shows that there is a negative correlation between falling and steps per minute, stride length and stride velocity. A positive correlation was shown between falling and stance duration and stance percentage. However, these physiological traits are all associated with old age, and old age is associated with falling, so the relation might be quite indirect. On the other hand, stride and stance as physiological aspects that can be measured with relative ease using a mobile phone.\cite{LLKgaitPatterns}