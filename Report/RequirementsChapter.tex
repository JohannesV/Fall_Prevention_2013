\chapter{Requirements}
The requirements for the different parts of the project were about the requirements for the application, the underlying model and content provider, and the documentation for the model and content provider. 

\section{General requirements}
The requirements presented by the customer at the beginning of the project can be summarized as follows:

\begin{itemize}
\item A model for fall risk assessment based on sensors in Android phones should be developed.
\item The model should be accessible through a Content Provider.
\item The model should be accompanied by a proof-of-concept demo application.
\item The product should be available as open source through the Apache 2.0 license, explained below.
\item Development should follow an incremental approach.
\end{itemize}

\section{Elaborating on the requirements}
The initial requirements were too vague to base an implementation on, so the group worked on specifying the requirements further, in cooperation with the customer and later the health experts. The list of more precise requirement is presented below:

\subsection{Functional requirements}
\begin{itemize}
\item A standard Android content provider should provide an interface to the generated risk data. The data model should contain enough details to be able give at least a basic predication of fall risk. The data model should include:
\begin{itemize}
\item Gait speed - One of two major predicators of fall risk
\item Gait variability - The other major predicators of fall risk
\item Step count - A general measurement of health and activity level.
\end{itemize}
\item An example application should the developed. The demo application should:
\begin{itemize}
\item Appear as a health assistant, not as an application for fall risk assessment, as the elderly are reluctant to admit they have a risk for falling, and would not use such an application.
\item Provide useful information for people without a computational background. This way the health people can assess the usefulness of the model without having to learn the computational details.
\end{itemize}	
\end{itemize}	

\subsection{Non-functional requirements}
\begin{itemize}
\item Development should be done in short iterations, and results should be presentable at each weekly customer meeting.
\item The demo application should be easy to use, even for elderly people unfamiliar with electronic equipment.
\item The application system should be modular and well-documented, so that it is easy to implement new applications based on the content provider, as well as connect new data sources to the content provider.
\end{itemize}	


\section{Understanding the requirements}

The requirements were specified after several meetings with the customer. The understanding the group had of the requirements before meeting the customer was not sufficient to create a rigorous set of requirements. Also, the requirements presented at the beginning of the project were vague, and needed more elaboration as the project progressed. \\

The meetings with the customer served their purpose, and filled in a lot of holes that were missing from the preliminary description for the following reasons:
\begin{itemize}
\item In the beginning the development process used an incremental approach. This essentially means that the requirements expands as the time goes, and to define the requirements from the beginning was therefore impossible.
\item Some of the requirements needed domain knowledge to be completely specified, and thus could not be set before the meeting with the health experts.
\item The customer expected weekly meetings with the whole group present, and that goals had been achieved before each meeting. At each meeting a new goal was set. Later, however, the development process changed, and longer term goals and requirements could be formulated. 
\end{itemize}

\section{Changes in Requirements}

The customer did usually only specify requirements that should be done for the current sprint, but the group were given a task to make some \footnote{user stories}. The group then formed some requirements from the user stories, which turned out to be too extensive given the time limit.

\begin{itemize} 
\item Contacts \& contact list - In the beginning the the group tried to make a contact list to notify relatives about the progress of the user. The reason this were excluded later on were simply that it was not needed to make the application work and would require too much work and time, that was better spent to develop the main function of the application. 
\item Notifications by E-mail \& Short Message Service - At the same period as the group tried to develop the contacts list they also tried to make it so others could receive the notification created by the user either by mail or SMS. The group discarded this idea, together with the contacts idea, as it was too complicated to do with their current experience at that time.
\end{itemize} 
\section{License}
The results of the project was intended to be released under the Apache 2.0 license. It is an open source license, which can be found  at the following web site: \url{http://www.apache.org/licenses/LICENSE-2.0.html}\\
Highlights of what can, cannot and must be done under the license is described here:\\
\textbf{Must be done}
\begin{itemize}
\item include copyright
\item include the license
\item state what changes has been made
\end{itemize}
\textbf{What can be done}
\begin{itemize}
\item use copies of the licensed work without paying
\item use the content with commercial products later
\item modify the licensed work as desired
\item the work can be distributed as desired
\end{itemize}
\textbf{Cannot do}
\begin{itemize}
\item trademark the licensed work
\end{itemize}
The specific terms can be found at the mentioned address, and has also been included in the appendix, see \ref{appendix:license}, along with License and Notice files.
