\chapter{Requirements}

The general target of the project is to explore the opportunities that are provided by standard smart phones running the Android Operation System for preventing falls by predicting risk levels, and develop a prototype system. The target groups for the system are:
\begin{itemize}
\item Elderly that are self-sufficient and relatively healthy, but have some concerns regarding their own stability. They benefit from the application as it provides feedback about their risk level.
\item Health personnel. The application can work as an indicator for fall risk among the patients, and perhaps provide insights into possible 
solutions and causes to the falling problem.
\item Concerned relatives.
\end{itemize}

One major problem that was pointed out by the health experts is that the first target group does not, in fact, exist. Falling is an enormous taboo, and even people with a history of falling tend to convince themselves that every fall is an individual exception. To cope with this, the application has to be marketed as a general health promoter when targeting the elderly, but run fall risk detection in the background. Targeting the other two groups, the application can still be marketed as originally planned. 

\section{Initial requirements}
The initial requirements originated largely from the customer, with some course requirements added in. The requirements presented by the customer at the beginning of the project can be summarized as follows:

\begin{itemize}
\item A model for fall risk assessment based on sensors in Android phones should be developed.
\item The model should be accessible through a Content Provider.
\item The model should be accompanied by a proof-of-concept demo application.
\item The product should be available as open source through the Apache 2.0 license, explained below.
\item Development should follow an incremental approach.
\end{itemize}

\section{Elaborating on the requirements}
The initial requirements were too vague to base an implementation on, so a requirements elicitation process was initiated, where the group in conjunction with the customer, and later the health experts, gradually elaborated upon the requirements. The process resulted in a list of more precise requirement, as presented below:

\subsection{Functional requirements}
\begin{itemize}
\item A standard Android content provider should provide an interface to the generated risk data. The data model should contain enough details to be able give at least a basic predication of fall risk. The data model should include:
\begin{itemize}
\item Gait speed - One of two major predicators of fall risk
\item Gait variability - The other major predicators of fall risk
\item Step count - A general measurement of health and activity level.
\end{itemize}
\item An example application should the developed that demonstrates the data model.
\end{itemize}	

\subsection{Non-functional requirements}
\begin{itemize}
\item Development should be done in short iterations, and results should be presentable at each weekly customer meeting.
\item The demo application should appear as a health assistant, not as an application for fall risk assessment, as the elderly are reluctant to admit they have a risk for falling, and would not use such an application.
\item The demo application should be easy to use, even for elderly people unfamiliar with electronic equipment. This includes localization support at least for Norwegian, few visual details, and simple feedback.
\item The demo application should provide more detailed information that is useful for health personnel without a computational background.
\item The application system should be modular and well-documented, so that it is easy to implement new applications based on the content provider, as well as connect new data sources to the content provider.
\end{itemize}

\section{Omitted requirements} \label{expert_meeting_requirement}
The customer did usually only specify requirements that should be done for the current sprint, but the group were given a task to make some user stories (see \ref{user_stories}. The group then formed some requirements from the user stories, which turned out to be too extensive given the time limit.

\begin{description} 
\item[Contacts and notifications] An initially desired function was to maintain a list of contacts and their contact information and the possibility to notify the contacts by e-mail or SMS, in order to accommodate the second target group. This requirement was excluded later on because it was difficult to determine which types of situations that should prompt notifications, and that it would require significant amounts of time to develop this non-essential functionality. Before the decision was made to exclude this functionality, a system for creating, importing and maintaining a list of contacts was developed, and exists in the source code, but is the related screens are not accessible in the application.
\item[Stride length] In addition to step count and gait parameters, \emph{stride length} is another predicator of fall risk that may be measurable using sensors that are available on Android: One approach could be to employ a location detector (for instance GPS) to track distance walked, and find step length as $$\frac{steps taken}{distance walked}$$However, the additional effort required to implement location awareness overshadowed the perceived predicative power of the additional predicator.
\item[Inactivity detection] Long periods of inactivity are bad for the general health, as well as strong indicators of fall risk. While some effort was put into studying the patterns related to sitting down and getting up to be used in the context of inactivity detection, time constraints and the need to focus on step detection forces this requirement to be excluded. 
\end{description} 

The requirements for the interface was altered by an informal usability test done with the medical experts. Their feedback is summarized here: 
\begin{itemize}
\item Immediate and summarized feedback was desired, as it would be more helpful than feedback later.
\item A focus on falling is not a selling point, try health or lifestyle instead.
\item Menu is not obvious to people without android experience.
\item Interface seems too complex for target groups.
\item Text might be too small.
\item Graph needs more contrast. 

\end{itemize}
\section{License}
The results of the project were to be released under the Apache 2.0 license, as requested from the customer. It is an open source license, which can be found  at their web site.\footnote{\url{http://www.apache.org/licenses/LICENSE-2.0.html}\\}
Highlights of what can, cannot and must be done under the license is described here:\\
\textbf{Must be done}
\begin{itemize}
\item Include copyright
\item Include the license
\item State what changes have been made
\end{itemize}
\textbf{What can be done}
\begin{itemize}
\item Use copies of the licensed work without paying
\item Use the content with commercial products later
\item Modify the licensed work as desired
\item The work can be distributed as desired
\end{itemize}
\textbf{Cannot do}
\begin{itemize}
\item Trademark the licensed work
\end{itemize}
The specific terms can be found at the mentioned address, and have also been included in the appendix, see \ref{appendix:license}, along with License and Notice files.
