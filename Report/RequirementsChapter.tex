\chapter{Requirements}
The requirements for the different parts of the project was about the requirements for the application, the underlying model and content provider, and the documentation for the model and content provider. 

\section{Initial requirements}

According to the project description, the following should be developed:
\begin{itemize}
\item A model of physical movements based on common movement sensors found in Android smart phones.
\item An Android content provider that stores and makes available the data in this model through an Application Programming Interface(API).
\item An example application that can visualize this data.
\end{itemize}

\section{Understanding the requirements}

The requirements were specified after several meetings with the customer. The understanding the group had of the requirements before meeting the customer was not sufficient to create a set of requirements. The requirements were also subject to change as the project progressed. \\

The meetings served their purpose, and filled in a lot of holes that was missing from the preliminary description:
\begin{itemize}
\item The development process in question uses a iterative approach. This essentially means that the requirements expands as the time goes, and to define the requirements from the beginning is impossible. This decision was made in collaboration with the customer.
\item The team at SINTEF got a wide range of experts to help the group with health-specific features, e.g. making the algorithms to recognize movements.
\item The customer expected weekly meetings with the whole group present, and that goals had been achieved before each meeting. At each meeting a new goal until was set.
\end{itemize}

\section{Functional requirements}

\section{Nonfunctional requirements}