\chapter{Requirements}
The requirements for the different parts of the project was about the requirements for the application, the underlying model and content provider, and the documentation for the model and content provider. 

\section{Initial requirements}

According to the project description, the following should be developed:
\begin{itemize}
\item A model of physical movements based on common movement sensors found in Android smart phones.
\item An Android content provider that stores and makes available the data in this model through an Application Programming Interface(API).
\item An example application that can visualize this data.
\end{itemize}

\section{Current requirements}
Several meetings with the customer and health experts led the initial requirements to be out of date and a little vague which led to new and updated requirements.

\subsection{Functional requirements}
\begin{itemize}
\item An Android content provider that stores and makes the data available available for other applications through an API.
\begin{itemize}
\item Measure gait speed.
\item Measure gait variability.
\item Measure steps.
\item Filter out noise in terms of bad and unnatural movements.
\end{itemize}
\item An example application that can visualize this data and represent them in a pedagogical manner.
\begin{itemize}
\item Ability to do self-tests
\begin{itemize}
\item Reaction speed
\item Stand up sit down capability
\item Balance tests
\end{itemize}
\end{itemize}
\end{itemize}	

\subsection{Nonfunctional requirements}
\begin{itemize}
\item 80\% of the elderly should manage to use the app within 10 minutes of education given that they have some prior knowledge of smartphones.
\item Midly visually impaired should be able to use the application.
\item Color blind should be able to use the application.
\end{itemize}	

\section{Understanding the requirements}

The requirements were specified after several meetings with the customer. The understanding the group had of the requirements before meeting the customer was not sufficient to create a set of requirements. The requirements were also subject to change as the project progressed. \\

The meetings served their purpose, and filled in a lot of holes that was missing from the preliminary description:
\begin{itemize}
\item The development process in question uses a iterative approach. This essentially means that the requirements expands as the time goes, and to define the requirements from the beginning is impossible. This decision was made in collaboration with the customer.
\item The team at SINTEF got a wide range of experts to help the group with health-specific features, e.g. making the algorithms to recognize movements.
\item The customer expected weekly meetings with the whole group present, and that goals had been achieved before each meeting. At each meeting a new goal was set.
\end{itemize}

\section{Testing}
The tests were done by the black-box principle of testing, meaning that all the tests consider is , with no attention given to the inner workings of the methods. Because of the team following an agile development method, it was possible to create and test test-cases in short order. The test-cases was therefore written with results included.
The tests are corresponding screens for the tests were as follows:
\begin{description}
\item[Name Entry] is associated with tests 1.x, both when program is started the first time, and when name-change is initiated from settings.
\item[Event Details Screen] is associated with tests 2.x.
\item[Statistics Screen] is associated with tests 3.x.
\item[Clear history] can be initiated from settings screen and is associated with tests 4.x.
\item[Related people Screen] is associated with tests 5.x. 
\end{description}

\begin{figure}[h]
\label{tab:testList}
\caption{List of tests for the application}
\begin{tabular}{|p{\textwidth /4}||p{\textwidth /4}|p{\textwidth /4}|p{\textwidth /4}|}
\hline
Test number & Input value& Actual Result& Expected Result \\
\hline
\hline 
1.1 & Any valid name & Hello: + name & Hello: + name \\ 
\hline 
1.2 & Empty name-box & Hello: &Hello:  \\ 
\hline 
2.1 & Remove event button clicked & Event removed & Event removed \\ 
\hline 
2.2 & Keep event button clicked & Event is kept & Event is kept \\ 
\hline 
3.1 & Timeslot set to 10 min or less & Data from correct timeslot displayed & Data from correct timeslot displayed \\ 
%\hline 
%3.2 & • & • & • \\ 
\hline 
4.1 & Clear History Button Clicked & History Cleared & History Cleared \\ 
\hline 
5.1 & Add existing contact button clicked & Contact shows up on appropriate screen  & Contact shows up on appropriate screen \\ 
\hline 
5.2 & Create New Contact function used & New Contact shows up on appropriate screen & New Contact shows up on appropriate screen \\ 
\hline 

\end{tabular} 
\end{figure}
