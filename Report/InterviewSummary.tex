\section{Summary of interview with medical professionals}

\subsubsection*{Feedback on Interface and Presentation of the program}
\begin{itemize}
\item Immediate and summarized feedback was desired, as it would be more helpful than feedback later.
\item A focus on falling is not a selling point, try health or lifestyle instead.
\item Menu is not obvious to people without android experience.
\item Interface seems too complex for target groups.
\item Text might be too small.
\item Graph needs more contrast. 

\end{itemize}

\textbf{Feedback on program functionality}
\begin{itemize}
\item Feedback should be tailored to particular groups.
\item Self-testing can be useful for reducing risk and keeping awareness.
\item Small tests to measure reaction time, for example reaction to light or sound.
\item Researchers could also be interested in data gathering.
\item Measuring transitions between sitting/lying and standing up.
\item Compare behavior on a weekly basis will give an overview of whether things are good or not.
\end{itemize}

\textbf{Feedback on medicinal stuff}
\begin{itemize}
\item Preventing inactivity over time is useful to reduce risk.
\item People with stability problems increase risk when moving much.
\item Variable gait pattern is useful for predicting risk.
\item Step time is more robust (and hopefully easy to measure).
\item A common exercise is to take a step forwards, sideways, backwards, sideways (measuring a box).
\item People in risk group are those who cut out walking, taking the bus, want help from others around the house.
\item Irregularity in speed and length is important to look for.
\item Appropriate movement amount for gender and age group is usually known.
\item Time needed to turn in place.
\end{itemize}