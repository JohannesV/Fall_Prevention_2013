\section{Related Applications}
There exist several other open source programs and commercial programs that solve related problems. These could function as inspiration for design, interfaces and functions, and open source programs can be imported and incorporated into the project. The most relevant related applications are presented below. Three of the applications listed below are based on Global Positioning System (GPS) signals to measure movements. The reason for this was because of the early focus on stride length. Later meetings with the health experts revealed this to not be as useful as first anticipated. Also the focus on movement was because of the core functionality to the project goals to measure movement in an efficient way.

\subsection{Endomondo Sports Tracker\footnote{This application can be found at \url{http://www.endomondo.com}}}
This is a popular app for exercising. It can be used to track routes, times and progress in training and activities. It uses GPS (and accelerometer) to find position on maps and track the location. Very easy to use, and automatically syncs up to endomondo.com. Has an inspiring design and sync-functionality.

The main part of Endomondo is the inspirational part to get people to move; you can compare yourself to others, e.g. friends, and you have a really easy way to see what an exercise is worth in amount of calories and amount of burgers burned by an exercise.

A lot of aspects from this app can be used during design, especially labeling and representation of data.

\subsection{GPS Status \footnote{This application can be found at \url{https://play.google.com/store/apps/details? id=com.eclipsim.gpsstatus2}}}
This application is used to view detailed status about the GPS system on the phone. It has a very advanced interface with lots of numbers to show information, but for a rookie user the data rarely translates into information.

The app succeeds to use the sensors very heavily.

\subsection{Pedometer\footnote{Application can be found at \url{https://code.google.com/p/pedometer/}}}
An open source app that has a GPLv3 license that is compatible with our Apache 2.0 license. The app uses the same sensors that may be useful for the current project. Taking advantage of this could reduce development cost drastically.

The creator's description of how the step detection algorithm works: \textit{``Basically, it aggregates the sensor values, finds the maximum and minimum, and if the difference is bigger than a value (which depends on the sensitivity setting) then it counts it as a step. There's is some additional optimization, which I arrived to through experimentation."}\footnote{Taken from the FAQ at \url{https://github.com/bagilevi/android-pedometer}}

\subsection{GPS Tracker\footnote{Application can be found at: \url{https://code.google.com/p/open-gpstracker}}}
This program adds the capability to store and review where you and your Android device have been. Basically you press record at the start of your trip and your phone stores the route you take. This route is drawn real-time on the Maps functionality of Android or in the background with an idle device. The route is stored on your phone for review and further use. The applications tracks location by GPS, hence the name. An accurate description of the program type would be a GPS logger.

It contains the ability to log location, something that might prove useful for current project. Since it is open source and has a compatible license (GNU v3), it can be incorporated easily.

