\chapter{Introduction}

Damage from falling is the main cause for hospitalization among the elderly. In addition to the enormous costs incurred to society, falling is also the cause of great personal tragedy. As the use of computational devices becomes more common, it seems natural to examine how such devices can contribute to preventing such accidents.

This project is a cooperation between SINTEF and a group of students at NTNU for the course IT2901. Its goal is to develop a model for risk level of movement based on common sensors found in Android smart phones, and build an API to make the model accessible for third-party developers.

\section{Group introduction}
The group consists of 5 members, four of which were from Norwegian University of Science and Technology, and the remaining student was an exchange student from Germany, Technical University of Dortmund. The members had experience with programming and project development with the programming languages Java and C$  $, but little experience with Android applications. 

\section{Stakeholder introduction}
There were several stakeholders in this project, and they shall be introduced here. 
\subsection{SINTEF-Customer}
The customer was "Selskapet for INdustriell og TEknisk Forskning ved norges tekniske hoegskole", or SINTEF for short. The person representing SINTEF was Babak A. Farshchian, Adjunct Associate Professor at NTNU.
\subsection{Supervisor- Tinna T\'{o}masd\'{o}ttir}
The developers were assigned a supervisor from the institute. The supervisor, Tinna T\'{o}masd\'{o}ttir, gave guidelines and feedback on how to build the reports and make documentation. She took part in grading the project. 
\subsection{Medical Experts}
Profs Jorun L. Helbostad and Beatrix Vereijken provided invaluable help with understanding questions from the medical domain. They were also associated with the FARSEEING project. 
\subsection{FARSEEING}