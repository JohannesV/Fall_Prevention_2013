\chapter{Introduction}

Damage from falling is the main cause for hospitalization among the elderly. In addition to the enormous costs incurred to society, falling is also the cause of great personal tragedy. As the use of computational devices becomes more common, it seems natural to examine how such devices can contribute to preventing such accidents.

This project is a cooperation between SINTEF and a group of students at NTNU, as a part of the European research project FARSEEING. The students participate as a part of the course IT2901.

A generic goal of the project is to explore potential uses for smart phones in fall risk assessment and prevention. Specifically, the goal is to develop a model for fall risk level based on movement detected by common sensors found in Android smart phones, and build an API to make the model accessible for third-party developers.

\section{FARSEEING}
"FARSEEING is a collaborative European Commission funded research project with 11 partners distributed in 7 EU countries. It aims to provide a thematic network focusing on the issue of promoting healthy, independent living for older adults. FARSEEING aims to promote better prediction and prevention of falls and to support older adults with a focus on ICT devices and the unique proactive opportunities they can provide to older adults to support them in their own environment."\footnote{Quote from the "About Us" section of the FARSEEING website \url{http://farseeingresearch.eu}}

\section{IT2901}
IT2901 - Informatics, Project II - is a course given at NTNU, where a group of students are given a project to work on independently over the course of a semester. The projects are provided by private and public enterprises in cooperation with NTNU. The university provides a supervisor to give the students feedback throughout the semester, but the students are responsible for direct communication with the customer. The course is worth 15 ECTS, which amounts to approximately 20 hours of work per week for each student. 

\section{Group introduction}
The group consists of 5 members, four of which were from Norwegian University of Science and Technology, and the remaining student was an exchange student from Germany, Technical University of Dortmund. The members had experience with programming and project development with the programming languages Java and C$  $, but little experience with Android applications. 

\section{Stakeholder introduction}
In addition to the developers, the following parties were stakeholders in the project.
\subsection{SINTEF-Customer}
The customer was "Selskapet for INdustriell og TEknisk Forskning ved norges tekniske hoegskole", or SINTEF for short. The person representing SINTEF was Babak A. Farshchian, Adjunct Associate Professor at NTNU.
\subsection{Supervisor- Tinna T\'{o}masd\'{o}ttir}
The developers were assigned a supervisor from the institute. The supervisor, Tinna T\'{o}masd\'{o}ttir, gave guidelines and feedback on how to build the reports and make documentation. She took part in grading the project. 
\subsection{Medical Experts}
Profs Jorun L. Helbostad and Beatrix Vereijken provided invaluable help with understanding questions from the medical domain. They were also associated with the FARSEEING project. 