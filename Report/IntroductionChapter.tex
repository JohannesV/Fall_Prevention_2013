\chapter{Introduction}

Damage from falling is the main cause for hospitalization among the elderly. In addition to the enormous costs incurred to society, falling is also the cause of great personal tragedy. As the use of computational devices becomes more common, it seems natural to examine how such devices can contribute to preventing such accidents.

This project is a cooperation between SINTEF and a group of students at NTNU for the course IT2901. Its goal is to develop a model for risk level of movement based on common sensors found in Android smart phones, and build a API to make the model accessible for third-party developers.

\section{Group introduction}
The group consists of 5 members, four of which were from Norwegian University of Science and Technology, and the remaining student was an exchange student from Germany (Insert school here). The members had experience with programming and project development with the languages java and c$  $, but little experience with Android applications. 

\section{Customer introduction}
The customer was "Selskapet for INdustriell og TEknisk Forskning ved norges tekniske hoegskole", or SINTEF for short. The person representing SINTEF was Babak A. Farshchian, Adjunct Associate Professor at NTNU. 