\chapter{Research and Alternate Solutions}
During the planning stage, the groups was able to identify two areas which required research:
\begin{description}
\item[Domain knowledge:] No member of the group had any familiarity with the application domain, neither with the specific domain (fall studies) nor any of the related domains (like healthcare, medicine or sport science). A basic understanding of the domain is critical for communication with the expert group, as well as for the ability to develop and assess solutions independently of the expert group.
\item[Related software:] Related applications function as sources of inspiration, proofs of what is feasible to implement or even as aids during programming. Open source software can be partially or entirely incorporated into the code. Knowledge of related software can thus increase development efficiency.  
\end{description}
For each of the two areas of research, one member of the group was assigned to research and compile a concise report, so that the rest of the group could efficiently attain the required knowledge.   

The two reports will be reproduced in the following sections.

\section{Falling: Causes, Consequences and how to avoid it}
This report collected information from several sources on the subject of falls among the elderly. Particular focus is given to risks and avoidance strategies.

\subsection{Risk factors}
There are several factors that predicate risk of falling. Typical risk factors are:
\begin{itemize}
\item 
\textbf{Biological risk factors (sorted by order of significance):  \cite{fallsRubenstein}}
\begin{itemize}
\item Muscle weakness
\item Balance deficit
\item Gait deficit
\item Visual deficit
\item Mobility limitation
\item Cognitive impairment
\item Impaired functional status
\item Postural hypertension
\end{itemize}
\item 
\textbf{Behavioral risk factors(Not in sorted order):}
\begin{itemize}
\item Inactivity \cite{cdcComProg}
\item Medications \cite{cdcComProg}
\item Alcohol use \cite{cdcComProg}
\item Living alone \cite{housing}
\end{itemize}
\item 
\textbf{Home/environmental risk factors(not in sorted order): \cite{WHO}}
\begin{itemize}
\item Bad footwear or clothing.
\item Dangers in the house or in public places.
\item Unfamiliarity with walking aids such as canes, crutches, walking chairs.
\end{itemize}
\end{itemize}

A fall is normally caused by the interaction of two or more of these risk factors, but home or environmental risk factors cause only 30-50\% of all falls\cite{fallsRubenstein, cdcComProg}. What this means is that more than half of all falls happen without any influence of environmental factors. Also, in most of these cases where external factors play a role, the fall is in reality caused by an interaction between these and physiological aspects. The second and third most common single causes of falling are gait/balance disorders and dizziness/vertigo, followed by \emph{drop attacks}, which are sudden falls without loss of consciousness or dizziness\cite{fallsRubenstein}. As the project task is to develop a model for physical movement, it might be necessary to down-prioritize identifying risk for dizziness/vertigo and drop attacks, as well as ignore external factors, in order to focus purely on physiological aspects, such as gait or balance.

Many older adults are unaware of their risk factors, and therefore unable to take preventive actions. Even older adults with a history of falling have normally been given little education about the potential risk factors. Any sort of risk assessment, is therefore be very beneficial, especially when the results are discussed with a healthcare provider\cite{cdcComProg}. This fact illustrates the usefulness of the planned application. 

It has been shown that many of the biological risk factors can be reduced effectively by preforming regular physical activities. Specifically strength, gait and balance has been shown to be improvable by simple exercise regimes - even for frail patients - in a number of studies\cite{LMTassessPrev, cdcComProg, WHO}. A potential focus area for the app can therefore be encouraging exercise and healthy lifestyles.

\subsection{Advice for prevention}
The individual can easily reduce the risk of falling greatly by taking certain measures. A list of measures normally, by the authorities, recommended for older people in the target group is given below\cite{cdcYouPrevent}:

\begin{itemize}
\item Begin a regular exercise program.
\begin{itemize}
\item Exercises that improve balance and coordination are the most helpful.
\item Is the only measure that by itself reduces the risk of falling independently of individual circumstances. 
\end{itemize}
\item Have your health care provider review your medicines, even over-the-counter ones.
\begin{itemize}
\item Avoid medicines that can make you dizzy or sleepy.
\end{itemize}
\item Have your vision checked.
\item Make your home safer.
\begin{itemize}
\item Remove or fasten small rugs and carpets.
\item Remove wires and cords away from commonly taken paths. Tape wires to the walls, and if necessary install an additional power outlet.
\item Keep items where you can reach them without having to climb. If you have to use a step stool, get a stable one with handrails.
\item Remove loose items that can make you trip (books, papers, clothes, etc.) from the floor and stairs.
\item Add grab bars and non-slip mats to the bathroom.
\item Make your home brighter.
\item Repair broken or uneven steps and handrails in the staircase.
\item Avoid using the staircase more than necessary, for instance by installing a light switch at the top as well as the bottom of the stairs. 
\item Wear shoes, even inside. Avoid walking barefoot or wearing slippers.
\end{itemize}
\end{itemize}

\subsection{Physiological aspects}

The list below explains the physiological factors that contribute to stability. “A marked deficit in any one of these factors may be sufficient to increase the risk of falling; however, a combination of mild or moderate impairments in multiple physiological domains also may increase the risk of falling. By directly assessing an individual's physiological abilities, intervention strategies can be implemented to target areas of deficit.” \cite{LMTassessPrev}

\begin{itemize}
\item Reaction time
\begin{itemize}
\item Hand
\item Foot
\end{itemize}
\item Vision
\begin{itemize}
\item Contrast sensitivity
\item Visual acuity
\end{itemize}
\item Vestibular function
\begin{itemize}
\item Visual field dependence
\end{itemize}
\item 
Peripheral sensation
\begin{itemize}
\item Tactile sensitivity
\item Vibration sense
\item Proprioception
\end{itemize}
\item Muscle force
\begin{itemize}
\item Knee flexion
\item Knee extension
\item Ankle dorsiflection
\end{itemize}
\end{itemize}


There exists an array of tests that can measure the performance on these aspects. Lord et al. \cite{LMTassessPrev} provides one possible set of tests. Some of these are possible to implement in an app, but none of them are based on hip movement.

The test developed by Lord et al. \cite{LMTassessPrev} was used to classify older adults into fallers or non-fallers, and has an accuracy of $75-80\%$ in different experiments. If a test which disregards hip movement patterns performs well, there may be reasons to believe that hip movement is only a minor factor in detecting fall risk. However, 

Walking behaviour seems to be generally associated with falling. Lord et al. \cite{LLKgaitPatterns} shows that there is a negative correlation between falling and steps per minute, stride length and stride velocity. A positive correlation was shown between falling and stance duration and stance percentage. However, these physiological traits are all associated with old age, and old age is associated with falling, so the relation might be quite indirect. On the other hand, stride and stance as physiological aspects that can be measured with relative ease using a mobile phone.

\section{Related Applications}
There exist several other open source programs and commercial programs that solve related problems. These could function as inspiration for design, interfaces and functions, and open source programs can be imported and incorporated into the project. The most relevant related applications will be presented below.

\subsection{Endomondo Sports Tracker}
This is a popular app for exercising. It can be used to track routes, times and progress in training and activities. It uses GPS (and accelerometer) to find position on maps and track the location. Very easy to use, and automatically syncs up to endomondo.com. Has an inspiring design and sync-functionality.

The main part of Endomondo is the inspirational part to get people to move; you can compare yourself to others, e.g. friends, and you have a really easy way to see what an exercise is worth in amount of calories and amount of burgers burned by an exercise.

A lot of aspects from this app can be used during design, especially labelling and representation of data.

\subsection{GPS Status}
This application is used to view detailed status about the GPS system on the phone. Has a very advanced interface with lots of numbers to show information, but for a rookie user the data rarely translates into information.

The app succeeds to use the sensors very heavily.

\subsection{Pedometer}\footnote{Application can be found at \url{https://code.google.com/p/pedometer/}}
An open source app that has a GPLv3 license that is compatible with our Apache 2.0 license. The app uses the same sensors that may be useful for the current project. Taking advantage of this could reduce development cost drastically.

The creator's description of how the step detection algorithm works\footnote{Taken from the FAQ at \url{https://github.com/bagilevi/android-pedometer}}: \textit{Basically, it aggregates the sensor values, finds the maximum and minimum, and if the difference is bigger than a value (which depends on the sensitivity setting) then it counts it as a step. There's is some additional optimization, which I arrived to through experimentation.}

\subsection{GPS Tracker}\footnote{Application can be found at: \url{https://code.google.com/p/open-gpstracker}}
This program adds the capability to store and review where you and your Android device have been. Basically you press record at the start of your trip and your phone stores the route you take. This route is drawn real-time on the Maps functionality of Android or in the background with an idle device. The route is stored on your phone for review and further use. The applications tracks location by GPS, hence the name. An accurate description of the program type would be a GPS logger.

It contains the ability to log location, something that might prove useful for current project. Since it is open source and has a compatible license (GNU v3), it can be incorporated easily.

\section{Lessons learned}
The research provided significant benefits for the group by providing a solid knowledge base for the application domain, so time spent on research was well spent. However, in retrospect the group was able to identify some faults in their research planning, and learned some lessons about how to avoid the same mistakes in the future:

\begin{itemize}
\item
Even though the research were accurate, the group was reluctant to employ the research results in the application before they were confirmed by the health experts. As the meeting with the health experts did not take place before 22.03, the group lost valuable time waiting, time that could have been spent for implementation. The lesson learned is to rely on research results, at least until a more reliable source of information is available (in this project, the expert group).
\item
Even though research was useful in the domains that were covered, some areas that could have benefited from research were not identified during planning, and were therefore not given thorough research. Because of this, some of the customer's demands were misinterpreted at first, forcing the group to change the application architecture at a later stage. In particular, the group was unfamiliar with the standards of Android development, especially the general architecture of Content Provider. A solid foundation in this area could have aided understanding. The lesson learned is to spend more time trying to identify potential research areas. 
\end{itemize}