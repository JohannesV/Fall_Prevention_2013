\chapter{Conclusion}
In this chapter will have information concluding the project, such as what the group learned and experienced.
\section{New Experiences}
All members of the group has learned much about teamwork, project management, planning, programming in a group, programming towards the Android platform, programming documents in LaTeX, and writing documentation for project activities. 
\section{Learning experiences from development and tool use}
Following are things that relate to project management and software development that the group learned during the project:
\begin{itemize}
 \item  It is important to ensure that documentation is done in a systematic fashion. In the beginning, a large amount of the reports and documentation that the group made for internal use, turned out to create more work and confusion later on. This was in part because it was not thought to be necessary at the time. Looking back, it seems like it would have been much more efficient to write reports well, and write them in LaTeX  while they were still new.
 \item Communication and what tools to use, was in flux for some time. Even when a few tools were settled on, the group still had problems with making sure all the members were on the same page regarding what needed to be done. 
 \item  Usage of tools which not all members were equally experienced with turned out to be problematic. This was because of misunderstanding and difficulties which could often be resolved by only one or two members, and slowed down the entire group. This might have been mitigated if all the members had a better understanding of the tools that were used. This could be accomplished by teaching sessions and group members being encouraged to seek a better understanding on their own.
 \end{itemize} 
 

\section{Lessons learned about Research and Prestudies}
The research provided significant benefits for the group by providing a solid knowledge base for the application domain, so time spent on research was well spent. However, in retrospect the group was able to identify some faults in their research planning, and learned some lessons about how to avoid the same mistakes in the future:

\begin{itemize}
\item
Even though the research were accurate, the group was reluctant to employ the research results in the application before they were confirmed by the health experts. As the meeting with the health experts did not take place before 22.03, the group lost valuable time waiting, time that could have been spent for implementation. The lesson learned is to rely on research results, at least until a more reliable source of information is available (in this project, the expert group).
\item
Even though research was useful in the domains that were covered, some areas that could have benefited from research were not identified during planning, and were therefore not given thorough research. Because of this, some of the customer's demands were misinterpreted at first, forcing the group to change the application architecture at a later stage. In particular, the group was unfamiliar with the standards of Android development, especially the general architecture of Content Provider. A solid foundation in this area could have aided understanding. The lesson learned is to spend more time trying to identify potential research areas. 
\end{itemize}