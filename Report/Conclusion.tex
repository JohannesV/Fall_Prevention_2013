\chapter{Conclusion}
In this chapter there is information concluding the project, such as what the group learned and experienced.
\section{Lessons learned}
The group learned some valuable lessons while working with the project. Both positive and negative aspects of working as a team and with unfamiliar tools were encountered as propagated in the following subsections.
\subsection{Development and tool usage}
Related to project management, tool usage and software development, the group gained the following experience:
\begin{itemize}
 \item  It is important to ensure that documentation is done in a systematic fashion. In the beginning, a large amount of the reports and documentation that the group made for internal use, turned out to create more work and confusion later on. This was in part because it was not thought to be necessary at the time. Looking back, it seems like it would have been much more efficient to write reports well, and write them in LaTeX  while they were still new.
 \item Communication and what tools to use, was in flux for some time. Even when a few tools were settled on, the group still had problems with making sure all the members were on the same page regarding what needed to be done. 
 \item  Usage of tools which not all members were equally experienced with turned out to be problematic. This was because of misunderstanding and difficulties which could often be resolved by only one or two members, and slowed down the entire group. This might have been mitigated if all the members had a better understanding of the tools that were used. This could be accomplished by teaching sessions and group members being encouraged to seek a better understanding on their own.
\end{itemize} 

\subsection{Research and prestudies}
The research provided significant benefits for the group by providing a solid knowledge base for the application domain, so time spent on research was time well spent. However, in retrospect the group was able to identify some faults in their research planning, and learned some lessons about how to avoid the same mistakes in the future:
\begin{itemize}
\item
Even though the research was accurate, the group was reluctant to employ the research results in the application before they were confirmed by the health experts. As the meeting with the health experts did not take place before 22.03.2013, the group lost valuable time waiting, time that could have been spent for implementation. The lesson learned is to rely on research results, at least until a more reliable source of information is available (in this project, the expert group).
\item
Even though research was useful in the domains that were covered, some areas that could have benefited from research were not identified during planning, and were therefore not given thorough research. Because of this, some of the customer's demands were misinterpreted at first, forcing the group to change the application architecture at a later stage. In particular, the group was unfamiliar with the standards of Android development, especially the general architecture of Content Provider. A solid foundation in this area could have aided understanding. The lesson learned is to spend more time trying to identify potential research areas. 
\end{itemize}

\subsection{New Experiences}
All members of the group learned much about teamwork, project management, planning, programming in a group, programming towards the Android platform, programming documents in LaTeX, and writing documentation for project- and research activities.

\section{Conflict handling}
There were no one-on-one conflicts, but some conflicts arose right after mid-term regarding work load and willingness to prioritize the project. To try to solve this the group leader communicated the issues and its potential solutions via e-mail, with no further response. Some group members took the e-mail seriously and stepped up their work load, but on the other hand, some members ignored this wake up call completely and did not take any action. A meeting with the supervisor was scheduled to try to get an idea to solve the situation, but was used for other purposes instead, because the group did not have an opportunity to discuss matters beforehand. Due to the lack of a proper meeting, the situation did not get resolved. Some of the team members still prioritized otherwise, while others worked harder.
Technical disagreements were solved by discussing the possible solutions with the entire group, until all the group members agreed on a course of action.

\section{Fulfilled requirements and further work}
In section \ref{elaboratingrequirements} on elaborating on the requirements, all the requirements are listed. Some requirements were met, some were not, and some only partly:
\begin{description}
\item[Content Provider] was made according to specifications and all the data specified was modelled and documented. Its design was based on standard content provider regulations according to Android guidelines, making it easy to implement in any Android app.
\item[Demonstration app] was made partly according to requirements.
\begin{itemize}
\item It did not have the ability to project all the data in an user friendly matter. However, the most important part of data, steps, were projected quite successfully in several ways.
\item The app did not appear as a health experts tool, but as an general easy to use health app, according to requirements.
\item It supports both Norwegian and English.
\item Unfortunately, the app did not have the ability to show detailed and useful information for health personnel without having some computational experience.
\end{itemize} 
\item[Development] was usually executed in short and intensive periods, to meet the weekly deadlines given by the customer.
\end{description}
Even though the project is finished, there will always be improvements to make, bugs to fix, and features to add. Some of them are listed below:
\begin{description}
\item[Fix bugs] discovered during acceptance testing.
\item[Research and improve] the Step Detector according to section \ref{step_detector_improvements} on Room for improvements. 
\item[Extend] the Step Detector to measure stride length with GPS.
\item[Extend] the Health Helper with the ability to do personal tests and measure inactivity elaborated in section \ref{expert_meeting_requirement}.
\item[Improve] the Health Helper by adding features suggested during acceptance testing.
\item[Research] critical thresholds for fall risk. The thresholds currently used are not based on empirical data.
\item[Research] better messages to display to users of the Health Helper. The messages shown now are too similar and makes little to none sense.
\item[Proper testing] throughout the system. The acceptance testing should be done by more than one person with one device to get diverse feedback and ensure compatibility with several devices.
\end{description}
