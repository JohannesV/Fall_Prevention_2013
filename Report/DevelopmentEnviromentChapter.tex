\chapter{Tools}


\section{Github}

\label{def:githubDev}It is requested by the customer that the group use the tool Github to share code and perform version control. Github has browser-based interfaces, download-able clients, and a robust command-line interface, meaning all the members of the group can make use of it. There were some problems associated with learning how to use and fix problems with it, as not all group members had experience with this tool. 

\section{Eclipse with Android Development Tools Plugin}
The demands from the Integrated Development Environment(IDE) are as following:
\begin{itemize}
\item Has support for programming applications for Android
\item Is understood by at least some members of the team
\end{itemize}
To fulfill these requirements, and because there are a plenty of tutorials that can be found, the group decides to use the Eclipse IDE, with add-on's for easier code and deploy towards Android. 

\section{Management Tools}
 
\begin{description}

\item[Trello]  \label{def:trello} is a collaboration tool with the ability to create interactive kanban-boards online. This was used to allow the group to coordinate tasks that were to be done, and the progress on the tasks, and which members were to work on which task. 

\item[It's learning] was used to distribute information that was not time -critical, with a message board being used. It's learning would not send messages when a new topic or message appeared, so it's use for time critical messages or making sure that everyone would read it was limited. 

\item[Email] was used for time critical communication, and for information that needed feedback swiftly, often within the same day.

\item[Github] \label{def:github} was used to share and synchronize code, and to describe and mark issues found in the code when problems were discovered. The relevant issues could then be discussed on the website.
Github is mostly used to update the followers of a repository and give them the newest version of the code, so it is  very ideal for coding in groups, where its users can simply push their finished work and the repository will automatically merge it with the already existing code, if done correctly. See chapter \ref{def:githubDev} for more.
\end{description}

\section{LaTeX}
LaTeX is used to write this report. LaTeX is a typesetting language, with support for varied formatting, including images and including other documents. It also has the option allowed for inclusion of multiple LaTeX documents (.tex documents). This leads to improved readability of LaTeX code and productivity, as several chapters could be altered at the same time.This functionality allows a group of multiple users to avoid the problems associated with multiple users writing in the same document. The editor used was Texmaker, an free latex-editor available on multiple platforms.
